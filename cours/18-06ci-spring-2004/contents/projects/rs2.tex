\documentclass[11pt]{amsart}


\newcommand{\R}{\mathbb R}
\newcommand{\Z}{\mathbb Z}
\newcommand{\la}{\langle}
\newcommand{\ra}{\rangle}

\newtheorem{theorem}{Theorem}
\newtheorem{definition}[theorem]{Definition}
\newtheorem{corollary}[theorem]{Corollary}
\newtheorem{remark}[theorem]{Remark}
\newtheorem{example}[theorem]{Example}

\begin{document}
\title{Abstract root systems}
\author{Your name here}
\date{\today}
\maketitle

%\thispagestyle{empty}

{\Large  18.099 - 18.06 CI.} 

{Due on Monday, May 10 in class.} 

\vspace{1cm} 

{\it Write a paper proving the statements formulated below. Add your own 
examples, asides and discussions whenever needed. }


Let $V$ be a Euclidean space, that is,
a finite dimensional real linear space with a symmetric 
positive definite inner product $\la, \ra$. 

Recall the definition of a reflection in $V$ (\cite{1}):

\begin{definition} A \emph{reflection} 
in $V$ with respect to a vector $\alpha \in V$ 
is defined by the formula: 
$$ s_\alpha (x) = 
x -\frac{2 \la x, \alpha \ra }{\la \alpha, \alpha \ra}\alpha. $$
\end{definition} 


\begin{definition} 
An \emph{abstract root system} in $V$ is a finite set $\Delta$ of nonzero 
elements of $V$ such that 
\begin{enumerate} 
\item{$\Delta$ spans $V$;}
\item{for all $\alpha \in \Delta$, the reflections  
$$ s_\alpha (\beta) = 
\beta -\frac{2 \la \beta, \alpha \ra }{\la \alpha, \alpha \ra}\alpha $$ 
map the set $\Delta$ to itself;}
\item{the number $\frac{2 \la \beta, \alpha \ra }{\la \alpha, \alpha \ra}$ 
is an integer for any $\alpha, \beta \in \Delta$.}
\end{enumerate} 
A \emph{root} is an element of $\Delta$. 
\end{definition} 

\begin{remark} This definition may seem weird to you, but wait till you 
see the examples. A golden rules in mathematics is that 
if you can define an object for which there are enough, but not too many 
nontrivial examples, it must have important implications. If a nice 
classification of such objects is available, it is a discovery. 
This indeed is the case with the abstract root systems. 
They  arise naturally in the theory of semisimple 
Lie algebras, and play an important role in a wide range of problems 
of algebra, representation theory and mathematical physics. 
At the same time, they can be defined and largely dealt 
with by means of linear algebra. This will be our goal.
\end{remark} 

\begin{example} Let $V$ be the following subspace of $\R^{n+1}$, $n \geq 1$:
$$ V = \{ \sum_{i=1}^{n+1} a_i e_i, \;{\rm with}\; \sum_{i=1}^{n+1}a_i =0 
\},$$ 
where $\{e_i\}_{i=1}^{n+1}$ is an orthonormal basis in $\R^{n+1}$, and all 
$a_i \in \R$. 
Check that the set $\Delta = \{e_i - e_j, i\neq j \}$, is an abstract 
root system. This is the root system of type $A_n$. 
Describe geometrically (sketch) the set $\Delta$ for $n=1$ and $n=2$.  
\end{example}   

\begin{example} Let $V$ be the space $\R^n$, $n \geq 2$ with an orthonormal 
basis $\{e_i\}_{i=1}^n$. Check that the set 
$$\Delta = \{ \pm e_i \pm e_j , i \neq j \} \cup \{ \pm e_i \} $$ 
is an abstract root system. This is the root system of type $B_n$.  
Describe geometrically the set $\Delta$ for $n=2$. 
\end{example} 

\begin{definition} An abstract root system is \emph{reducible} if it can be 
represented as a disjoint union of two abstract root systems 
$\Delta = \Delta' \cup \Delta''$, and each element of $\Delta' $ 
is orthogonal to each element of $\Delta''$. We say that $\Delta$ 
is \emph{irreducible} if it admits no such decomposition.  
\end{definition} 

\begin{example} \label{R^2}
Check that the root systems of types $A_2$ and $B_2$ are 
irreducible. Check that a union of two root systems of type $A_1$ 
is a reducible root system in $V = \R^2$, and sketch it. 
\end{example} 

We would like to classify the abstract root systems in any given dimension,
but to do this we need to be precise about what it means for two root
systems to be geometrically equivalent - we don't want to distinguish
between a root system and the one obtained from it by a simple process like
rotation or dilation.  With this in mind, say that two root systems 
$\Delta_1$ and $\Delta_2$ are 
\emph{isomorphic} if there is a linear automorphism
$f: V \to V$ taking $\Delta_1$ to $\Delta_2$,
such that for any roots $\alpha$ and $\beta$ in $\Delta_1$, 
$\frac{2 \la \beta, \alpha \ra }{\la \alpha, \alpha \ra} =
\frac{2 \la f(\beta), f(\alpha) \ra }{\la f(\alpha), f(\alpha) \ra}$.
We will classify the abstract root systems ``up to isomorphism'', 
that is, treating isomorphic root systems as being the same.

To get an idea of how to make this classification, we start by 
proving some elementary properties. 

\begin{theorem} \label{r1}
Let $\Delta$ be an abstract root system in $V$. 
\begin{enumerate} 
\item{If $\alpha \in \Delta$, then $-\alpha \in \Delta$.}
\item{If $\alpha \in \Delta$ and $\frac{1}{2}\alpha$ is not in $\Delta$, 
then the only possible elements of $\Delta \cup \{0\}$ proportional to 
$\alpha$ are $\pm \alpha$, $\pm 2 \alpha$ and $0$. } 
\item{If $\alpha$ is in $\Delta$ and $\beta \in \Delta \cup \{0\}$, 
then 
$$ \frac{2\la \beta, \alpha\ra}{\la \alpha, \alpha \ra } = 
0, \pm1, \pm2, \pm 3, \; {\rm or}\; \pm 4 , $$
and $\pm 4$ can occur only if $\beta = \pm 2 \alpha$. } 
\end{enumerate} 
\end{theorem} 
Hint: use the Cauchy-Schwarz inequality in $V$ (\cite{1}) to prove (3). 
 
\begin{example} Compute the numbers 
$n(\alpha, \beta) := 
\frac{2\la \beta, \alpha\ra}{\la \alpha, \alpha \ra }$ in two-dimensional 
examples of types $A_2$, $B_2$ and $A_1 \oplus A_1$ considered in Example 
\ref{R^2}. 
Does this exhaust the possibilities pridicted by the theorem? Can you guess 
any of the missing two-dimensional root systems?    
\end{example} 

Here is another statement to help in finding the root systems: 

\begin{theorem} Let $\Delta$ be an abstract root system in $V$. 
\begin{enumerate} 
\item{If $\alpha$ and $\beta$ are in $\Delta$, and 
$\la \alpha, \beta \ra >0$, then $\alpha - \beta$ is a root or $0$. 
If  $\la \alpha, \beta \ra <0$, then $\alpha + \beta$ is a root or $0$. }
\item{If $\alpha \in \Delta$ and $\beta \in \Delta \cup \{0\}$, 
then the set of elements of $\Delta \cup \{0\}$ of the form 
$\beta + n \alpha$, $ n \in \Z$,  contains all and only such elements with 
$-p \leq n \leq q$, for some $p \geq 0$ and $q \geq 0$ such that
$p-q = \frac{2 \la \beta, \alpha \ra}{\la \alpha, \alpha \ra}.$ } 
\end{enumerate} 
\end{theorem} 

Hint: To prove (2), assume there is a gap in the set of elements 
of $\Delta \cup \{0\}$ of the form $\beta + n \alpha$ and use (1) 
to get a contradiction. 


What is the maximal number of roots contained in a set $\beta + n \alpha$, 
where $\alpha \in \Delta$ and $\beta \in \Delta \cup \{0\}$? (Use both 
theorems above). 

Now  use the Euclidean geometry.  Recall that 
with the standard inner product in $\R^n$, the number 
$\la \alpha, \alpha \ra = || \alpha ||^2$ 
is the square of the length of the vector, and $ n(\alpha, \beta)$ 
can be written as  
$$ n(\alpha, \beta) = 2 \frac{||\beta ||}{||\alpha||} \cos(\phi), $$
where $\phi$ is the angle between $\alpha $ and $\beta$. 

Then we have 
$$ n(\alpha, \beta) \cdot n(\beta, \alpha) = 4 \cos^2(\phi).$$ 

Apply Theorem \ref{r1} to 
list the possibilities for angles $\phi$ and relative lengths 
between two nonproportional 
elements of an abstract root system. 

Using the above classification, construct examples of abstract root systems in 
$V= \R^2$.   
Remark: The one that contains two vectors with the angle between them 
$\phi=\pi/6$ is called $G_2$. 


\begin{thebibliography}{2}

\bibitem[1]{1} Your classmate, {\it Reflections in a Euclidean Space}, 
preprint, MIT, 2004. 


\end{thebibliography}


\end{document}


























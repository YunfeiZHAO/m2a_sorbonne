\documentclass[11pt]{amsart}

\usepackage{amsfonts}
\usepackage{amsmath}
\newcommand{\field}[1]{\mathbb{#1}}
\newcommand{\ud}{\mathrm{d}}
\renewcommand{\labelenumi}{(\alph{enumi})}

\begin{document}

\title{18.099/18.06CI - Homework 2}
\author{Juha Valkama}
\date{March 15, 2004}
\maketitle

\section*{Problem 1.}
Let $S$ be the space of all homogeneous polynomials of degree $p$ in
$n$ variables. Its elements can be expressed as sums of terms of form:
$P_m=\prod_{i=1}^n \alpha_i^{k_i}, \sum_{i=1}^n k_i = p.$ In order to
solve for the dimension of the space of all such polynomials we need
to find the largest possible number of distinct terms. This can be
reduced to finding the number of distinct non-negative integer valued
solution vectors $\{k_i\}$ to $\sum_{i=1}^{n} k_i = p\,$ due to the
uniqueness of factorization. From elementary combinatorics, the number
of positive, $k_i > 0$, solution vectors is $\binom{p-1}{n-1}$. To
obtain the number of non-negative solutions, $k_i \geq 0$, we note
that the number of such solutions is the same as the number of
positive solutions to $\sum_{i=1}^{n} k'_i=p+n$, where $k'_i=k_i+1$.
Hence, we obtain $\binom{p+n-1}{n-1}$ distinct non-negative solution
vectors.

Considering the set $U$ of all such distinct non-zero monomial
terms we find that they are linearly independent: $\sum c_m
P_m=0,\ P_m \in U,\ P_i\neq P_j, \, i \neq j \, \Rightarrow
\textrm{all } c_m = 0.$ Because $U$ is a maximal linearly
independent set of elements from $S$ we can consider its elements
as a basis for S. The dimension of S is simply the number of
elements in its basis and hence dim $S=\binom{p+n-1}{n-1}$. We
further note that $\binom{a}{b}>0$ for
 $a\geq 0 , \ a \geq b$ and thus for the cases $p=0,n=1$;
$p\leq n;p>n$ our formula works as expected.

\section*{Problem 2.}
Two finite dimensional linear spaces $L$ and $M$ are
isomorphic if and only if for $l = \textrm{dim } L, m =
\textrm{dim } M, \ l=m.$ To prove this let $\{u_i\}_{i=1}^{l}$ be
a basis for L and $\{v_j\}_{j=1}^m$ be a basis for M. A linear map
$f : L \mapsto M$ can be defined by $u_i \mapsto \sum_{j=1}^{m}
a_{ij} v_j.$ We consider such a map as a system of linear
equations, and assert from the properties of matrices that for
such map to be invertible, that is for it to be a one-to-one and
onto map, it is necessary and sufficient that the matrix of
coefficients $(a_{ij})$ be invertible. This is only possible if
$m=l$. Given $m=l$ we can choose $I_m$ as the coefficient matrix
and note that it is its own inverse. Thus, $L$ and $M$ are
isomorphic if and only if $\textrm{dim } L = \textrm{dim } M$.

\end{document}
